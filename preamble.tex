% E. Dunham -- Resume
% Contents Copyright (C) 2014 - 2016, E. Dunham

% LaTeX code for rendering the resume is distributed under the MIT license.
% See LICENSE.txt. It means you can use the code for whatever you want,
% including your own resume, but I'm not liable if it catches your computer on
% fire.

% Template originally developed by E. Dunham
% https://github.com/edunham/resume/blob/master/resume.tex
\usepackage[normalem]{ulem} % for the underlines
\usepackage[compact]{titlesec} % Shrink default spacings
\usepackage{tabto} % For aligning skills section
\usepackage{multicol} % for multicols command
\usepackage{ragged2e} % for /justify
\usepackage{hyperref}
\usepackage[none]{hyphenat}
\textwidth 7in
\textheight 10in
\topmargin -1in % Reclaim the default whitespace from top of page
\oddsidemargin -.25in % Reclaim whitespace on left, make it look centered
\pagenumbering{gobble} % Don't number pages
\setlength{\parindent}{0pt} % Don't indent paragraphs

\newcommand{\heading}[1]{
%    \section*{\centering\uline{\hfill #1 \hfill }} % Center the headings
    \section*{\uline{\hfill #1 }} % Right-align the headings
}
\newcommand{\squish}{
    \setlength{\parskip}{\whitespaciness pt}
}
\newcommand{\when}[1]{ % naming this 'date' would conflict with builtins
    \typeoffill \texttt{ #1}
}
\newcommand{\experience}[3]{ % place, optional title, date
    \ifx&#2&
        \item[{#1}]
    \else
        \item[{#1}, \emph{#2}]
    \fi
    \when{#3}
}
\newcommand{\event}[4]{
    \bf{#1} \tabto{2in} \texttt{#2} \tabto{3in} \normalfont
    \ifx&#3& \else
       \emph{ {#3},}
    \fi
    ``{#4}''
}
\newcommand{\contact}[4]{
    \centerline{ \large \texttt{ #1 $\bullet$ #2 $\bullet$ #3 }}
    \centerline{ \emph{ #4  $\bullet$ R\'{e}sum\'{e} current as of \today}}
}
\newcommand{\skill}[2]{
    \textbf{#1} \tabto{2.5in} #2
}
% Write C++ all fancy-like
% http://www.parashift.com/c++-faq-lite/latex-macros.html
\newcommand{\CPP}{
    C\hspace{-.05em}\raisebox{.4ex}{\tiny\bf +}\hspace{-.10em}\raisebox{.4ex}{\tiny\bf +}
}

% alternative to \skill, for extended lists of skills
% columnsep can be used here to unbalance the columns, with a negative number
% increasing the size of the right column versus the left.  '0cm' or equivalent
% will keep them balanced
%
% params: columnsep, heading, individual skills
\newcommand{\skillz}[3]{
    \vspace{-0.5cm}
    \squish
    \setlength{\columnsep}{#1}
    \begin{multicols}{2}
    \squish
    \RaggedRight % force to the hard left of the column
    \small
    \textbf{#2}
    \columnbreak
    \squish
    \justify
    \small
    #3
    \end{multicols}
    \vspace{-0.2cm} % yes, really necessary to keep this self-contained
}

% based on https://tex.stackexchange.com/a/148803
% intended to increase readability of longer entries under 'experience'
% params: indent/margin, item separation, top separation
%\newenvironment{hangingparlist}[3]
%    {\begin{list}
%        {}
%        {\setlength{\itemindent}{-#1}%%
%        \setlength{\leftmargin}{#1}%%
%        \setlength{\itemsep}{#2}%%
%        \setlength{\parsep}{#2}%%
%        \setlength{\topsep}{#3}%%
%        }
%    \setlength{\parindent}{-#1}%%
%    \item[]
%    }
%    {\end{list}}

% Projects/Contributions
% params: name, optional website, description
\newcommand{\project}[3]{
    \begin{description}
        \setlength{\parsep}{1em}
        \ifx&#2&
            \item[#1] --- #3
        \else
            \item[#1]\emph{#2} --- #3
        \fi
    \end{description}
}
